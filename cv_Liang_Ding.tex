%%%%%%%%%%%%%%%%%%%%%%%%%%%%%%%%%%%%%%%%%
% Medium Length Professional CV
% LaTeX Template
% Version 2.0 (8/5/13)
%
% This template has been downloaded from:
% http://www.LaTeXTemplates.com
%
% Original author:
% Trey Hunner (http://www.treyhunner.com/)
%
% Important note:
% This template requires the resume.cls file to be in the same directory as the
% .tex file. The resume.cls file provides the resume style used for structuring the
% document.
%
%%%%%%%%%%%%%%%%%%%%%%%%%%%%%%%%%%%%%%%%%

%----------------------------------------------------------------------------------------
%	PACKAGES AND OTHER DOCUMENT CONFIGURATIONS
%----------------------------------------------------------------------------------------

\documentclass{resume} % Use the custom resume.cls style

\usepackage[left=0.75in,top=0.6in,right=0.75in,bottom=0.6in]{geometry} % Document margins
\usepackage[colorlinks=true, urlcolor=blue, linkcolor=red]{hyperref}

\name{Liang (Adam) Ding} % Your name
\address{511 Live Oak Ave Unit 4, Arcadia, CA 91006} % Your address
%\address{123 Pleasant Lane \\ City, State 12345} % Your secondary addess (optional)
\address{(956)~$\cdot$~802~$\cdot$~8972 \\ adamdingliang@gmail.com} % Your phone number and email

\begin{document}
%----------------------------------------------------------------------------------------
%	WORK EXPERIENCE SECTION
%----------------------------------------------------------------------------------------

\begin{rSection}{Experience}
\begin{rSubsection}{Spatial Genomics, Inc.}{March 2024 - Present}
{Senior Bioinformatics Scientist}{Pasadena, CA}
\item Collaborated with assay development scientists to leverage spatial information to improve cell type identification for human solid tumors and mouse liver MASH. Identified disease-associated spatial niches and cell-cell communications between immune cell subpopulations and tumor cells in the tumor microenvironment.

\item Led the development of a novel algorithm to optimize barcode gene selection and distribution using reference bulk and/or scRNA-seq data to avoid the optical crowding issue in decoding the barcode genes from raw images. Provided detailed steps and guidance for software engineers to implement the algorithm.

\item Led the development of \href{https://tutorials.spatialgenomics.com/intro.html}{Spatial Genomics tutorials} for clustering, annotation, integration, and imputation of \href{https://spatialgenomics.com/product/}{GenePS} data from customers.

\item Created business brochures and posters to showcase the applications of GenePS data.
\end{rSubsection}

\begin{rSubsection}{Spatial Genomics, Inc.}{September 2022 - February 2024}
{Bioinformatics Scientist}{Pasadena, CA}
\item Spatial data analysis and visualization, such as the \href{http://kidneyviewer.spatialgenomics.com/}{mouse kidney Minerva story}, served as a showcase for the launch of the GenePS platform. 

\item Designed a novel deep degenerative algorithm for spatial niche alignment across ROIs by integrating spatial, image, and expression information from cells' spatial neighborhoods. Developed tools for mis-assigned transcript correction due to imperfect cell or nuclei segmentation.

\item Standard and customized gene panel designs, as well as probe designs for both internal and external use.

\item Comparative analysis of data from commercial in-situ platforms to guide the assay and 2nd generation instrument developments.
\end{rSubsection}


\begin{rSubsection}{St. Jude Children's Research Hospital}{December 2018 - September 2022}
{Bioinformatics Research Scientist}{Memphis, TN}
\item Oversaw and led the development of systems biology algorithms and software for scRNA-seq analysis, spatial and single-cell transcriptomics integration, and gene network reconstruction.

Systems biology tools and algorithms:\\
\href{https://www.nature.com/articles/s41467-025-59620-6}{scMINER} is a python/R package designed for preprocessing, QC, clustering, network inference, hidden driver analysis, and visualization of scRNA-seq analysis.\\
\href{https://www.nature.com/articles/s41467-023-38335-6}{NetBID2} is an R package for data-driven network-based Bayesian inference of drivers.\\
\href{https://github.com/jyyulab/SJARACNe}{SJARACNe} is a scalable software tool for reverse engineering gene networks.

\item Applied systems biology approaches to identify biomarkers and therapeutic targets for cancer treatment.

\item Manuscript writing and assisting with grant proposal drafting.
\end{rSubsection}


\begin{rSubsection}{St. Jude Children's Research Hospital}{July 2016 - December 2018}
{Senior Software Engineer}{Memphis, TN}
\item Development of bioinformatics workflow infrastructure for both research and clinical applications.

\item Analyzed genomics sequencing data from the clinical labs and ultimately provided feedback to impact patient care.

\item Collaborated with scientists to develop algorithms and software for analyzing multi-omics data.
\end{rSubsection}

%------------------------------------------------

\begin{rSubsection}{University of Georgia, Athens, Georgia}{September 2011 - June 2016}
{Graduate Instructor}{Athens, Georgia}
\item Lectured, oversaw in-class activities, and created all tests, quizzes, labs, and projects for a fast-paced system programming course that held approximately 30 students.

\textit{Lab Instructor and Teaching Assistant}
\item Taught Java programming and Eclipse.
\item Designed and graded lab assignments and projects.
\item Explained complex concepts in small groups and held office hours for individual student discussion.
\end{rSubsection}

\end{rSection}

%----------------------------------------------------------------------------------------
%	EDUCATION SECTION
%----------------------------------------------------------------------------------------

\begin{rSection}{Education}
{\bf University of Georgia, Athens, Georgia} \hfill {\em July 2016} \\ 
Ph.D., Computer Science
%Minor in Linguistics \smallskip \\
%Member of Eta Kappa Nu \\
%Member of Upsilon Pi Epsilon \\
%Overall GPA: 5.678

{\bf University of Texas - Pan America, Edinburg, Texas} \hfill {\em August 2011} \\ 
M.S., Computer Science

{\bf Zhengzhou University, Henan, China} \hfill {\em December 2008} \\ 
M.S., Applied Mathematics\\
B.S., Applied Mathematics
\end{rSection}

%----------------------------------------------------------------------------------------
%	TECHNICAL STRENGTHS SECTION
%----------------------------------------------------------------------------------------

\begin{rSection}{Technical Strengths}

\begin{tabular}{ @{} >{\bfseries}l @{\hspace{6ex}} l }
Operating Systems & Mac OS X, Windows, Linux \\
Computer Languages & Python, Rust, R, Bash, C++/C, SQL, Latex, Javascript \\
Databases & PostgreSQL, SQLite \\
Tools & PyTorch, Nvidia RAPIDS, Nextflow, Snakemake, Git, Docker \\
Single-cell/Spatial analysis & Scanpy, Seurat, scvi-tools, Squidpy, SpatialData, Minerva\\
Image analysis & ImageJ/Fiji, OpenCV, Baysor, Cellpose\\
\end{tabular}

\end{rSection}

%----------------------------------------------------------------------------------------
%	Publications
%----------------------------------------------------------------------------------------

\begin{rSection}{Publications}
* First authors contributed equally.  $^{\#}$ Corresponding authors. Sort by year in descending order.

\textbf{Peer Reviewed Journals}
\begin{enumerate}
\item Pan Q.*, \textbf{Ding L.}*, Hladyshau S., Yao X., et al., Chi H., Yu J.$^{\#}$ \href{https://github.com/jyyulab/scMINER}{scMINER}: a mutual information-based framework for clustering and hidden driver inference from single-cell transcriptomics data. Nature Communications 16, 4305, 2025.

\item McCastlain K.*, Welsh C.*, Ni Y.*, \textbf{Ding L.}*, et al., Yu J., Pounds S., Kundu M.$^{\#}$ Somatic mitochondrial DNA mutations are a source of heterogeneity among primary leukemic cells. Science Advanced, 2025.

\item Huang X.*, Li Y.*, Zhang J.*, Yan L., Zhao H., \textbf{Ding L.}, et al., Yu J.$^{\#}$, Yang J.$^{\#}$ Single-cell systems pharmacology identifies development-driven drug response and combination therapy in B cell acute lymphoblastic leukemia. Cancer cell, 42(4), 2024.

\item Yan K.*, Condori J., Ma Z., \textbf{Ding L.}, Dhungana Y., et al., Gottschalk S.$^{\#}$, Yu J.$^{\#}$ Integrome signatures of lentiviral gene therapy for SCID-X1 patients. 9(40), 2023.

\item Zhang Y.*, Pool A.H., Wang T., Liu L., Kang E., Zhang B., \textbf{Ding L.}, Frieda K., Palmiter R., Oka Y.$^{\#}$ Parallel neural pathways control sodium consumption and taste valence. Cell, 186(26), 2023.

\item Dong X.*, \textbf{Ding L.}, et al., Chi H., Zhang J., Yu J.$^{\#}$ \href{https://github.com/jyyulab/NetBID}{NetBID2} provides comprehensive hidden driver analysis. Cell, 14(1), 2023.

\item Chang D.*, \textbf{Ding L.}*, Malmberg R., Robinson D, Wicker M., Yan H., Martinez A., Cai L.$^{\#}$ \href{https://www.sciencedirect.com/science/article/pii/S277241582200013X}{Optimal learning of Markov $k$-tree topology}. Journal of Computational Mathematics and Data Science, 4(100046), 2022.

\item Shi H.*, Yan K.*, \textbf{Ding L.}, Qian C., Chi H., Yu J.$^{\#}$ Network Approaches for Dissecting the Immune System. iScience 23(8), 2020.

\item Silveira A.B., Kasper L.H., Fan Y., et al., \textbf{Ding L.}, Zhang J., Finkelstein D., et al., Baker S.J.$^{\#}$ H3.3 K27M depletion increases differentiation and extends latency of diffuse intrinsic pontine glioma growth in vivo. Acta Neuropathol. 137(4): 637-655, 2019.

\item Kohei K.*, \textbf{Ding L.}, Michael N.E., Stephen V.R., Scott N., John E., Juncheng D., Soheil M., Rhonda E.R., Michael R., Zhang, J.$^{\#}$ \href{https://github.com/stjude/RNAIndel}{RNAIndel}: discovering somatic coding indels from tumor RNA-Seq data. Bioinformatics, btz753, 2019.

\item Silveira A.B.*, Kasper L.H., Fan Y., Jin H., Wu G., Shaw T., Zhu X., Larson J.D., Easton J., Shao Y., Yergeau D.A., Rosencrance C., Boggs K., Rusch M.C., \textbf{Ding L.}, Zhang J., et al., Zhang J., Baker S.J.$^{\#}$ H3.3 K27M depletion increases differentiation and extends latency of diffuse intrinsic pontine glioma growth in vivo. Acta Neuropathologica. 137(4):637-655, 2019.

\item Xu K.*, \textbf{Ding L.}, Chang T.C., et al., Baker S.J., Wu G.$^{\#}$ Structure and evolution of \href{https://link.springer.com/article/10.1007/s00401-018-1912-1}{double minutes} in diagnosis and relapse brain tumors. Acta Neuropathologica. 137(1):123-137, 2019.

\item Alexander T.B.*, Gu Z.*, Iacobucci I.*, Dickerson K., Choi J.K., Xu B., et al., \textbf{Ding L}, Liu Y., Zhang J., et al., Inaba H.$^{\#}$, Mullighan C.G.$^{\#}$ The genetic basis and cell of origin of mixed phenotype acute leukaemia. Nature, 562(7727):373-379, 2018.

\item Mohebbi M.*, \textbf{Ding L.}, Malmberg R.L., Momany C., Rasheed K., Cai L.$^{\#}$ Accurate prediction of human miRNA targets via graph modeling of miRNA-target duplex. Journal of Bioinformatics and Computational Biology, 7:1850013, 2018.

\item \textbf{Ding L.}*, Xue X., LaMarca S., Mohebbi M., Samad A., Malmberg R.L.$^{\#}$, Cai L.$^{\#}$ Accurate Prediction of \href{https://academic.oup.com/bioinformatics/article/31/16/2660/321332}{RNA Nucleotide Interactions} with Backbone $k$-Tree Model. Bioinformatics, 31(16): 2660-2667, 2015.
\end{enumerate}


\textbf{Conference Proceedings}
\begin{enumerate}
\item \textbf{Ding L.}*, Xue X., LaMarca S., Mohebbi M., Samad A., Malmberg R.$^{\#}$, and Cai L.$^{\#}$ (2014) Abinitio prediction of RNA nucleotide interactions with backbone k-tree model, Proceedings of ECCB'14 Workshop on Computational Methods for Structural RNAs, Strasbourg France, 25-42.

\item \textbf{Ding L.}*, Samad A., Xue X., Huang X., Malmberg R.$^{\#}$, and Cai L.$^{\#}$ (2014) Stochastic k-tree grammar and its application in bimolecular structure modeling, International Conference on Language and Automata Theory and Applications (LATA 2014) Vol 8370, 308-322.

\item \textbf{Ding L.}*, Robertson J., Malmberg R.$^{\#}$, and Cai L.$^{\#}$ (2013) Protein closed loop prediction from contact probabilities, International Symposium on Bioinformatics Research and Applications (ISBRA 2013), 199-210.

\item \textbf{Ding L.}, Fu B., and Zhu B. (2011) Minimum Interval Cover and Its Application to Genome Sequencing. International Conference on Combinatorial Optimization and Applications (COCOA 2011), 287- 298.

\item \textbf{Ding L.}, Fu B., and Fu Y. (2010) Improved Sublinear Time Algorithm for Width- bounded Separators. International Workshop on Frontiers in Algorithmics, 101-112.

\item \textbf{Ding L.}, Fu B., Fu Y., Lu Z., and Zhao Z. (2010) $O((log n)^2)$ Time Online Approximation Schemes for Bin Packing and Subset Sum Problem. International Workshop on Frontiers in Algorithmics, 250-261.
\end{enumerate}


\textbf{Book Chapters}
\begin{enumerate}
\item Cai L., \textbf{Ding L.}, Huang X., Malmberg R.L., and Xue X. (2014) Stochastic grammar systems for biomolecular structure modeling, Integrative Bioinformatics for Biomedical Research: A No Boundary Thinking Approach, Huang and Moore ed. Cambridge Press, to appear.

\item \textbf{Ding L.}, Fu B. (2013) Algebrization and Randomization Methods. Handbook of Combinatorial Optimization, Springer, pp 171-220.

\item \textbf{Ding L.}, Fu B., Fu Y., Wan Y. (2012) Application of Width-Bounded Separators to Protein Side-Chain Packing Problem, Sequence and Genome Analysis: Methods and Applications, iConcep Press.
\end{enumerate}


\textbf{Preprints}
\begin{enumerate}
\item Michael R., \textbf{Ding L.}, Sasi A., Andrew T., Hongjian J., Michael M., Lawryn K., Andre S., Michael A.D., Suzanne J.B., Zhang J.. XenoCP: Cloud-based BAM cleansing tool for RNA and DNA from Xenograft. bioRxiv 843250.

\item \textbf{Ding L.}, Samad, A., Xue, X., Huang, X., Cai, L. (2013) Polynomial kernels collapse the W-hierarchy. arXiv preprint arXiv:1308.3613.
\end{enumerate}
\end{rSection}

%----------------------------------------------------------------------------------------

%----------------------------------------------------------------------------------------
%	HONORS and AWARDS
%----------------------------------------------------------------------------------------
\begin{rSection}{Honors and Awards}
\begin{itemize}
\item Ovation Silver Award, St. Jude Children’s Research Hospital, 2017
\item Dissertation Completion Award, Graduate School, University of Georgia, Athens, Georgia, 2015
\item Outstanding Graduate Teaching Assistant Award,  Graduate School, University of Georgia, Athens, Georgia, 2015
\item CUDA and GPU Programming Certificate CUDA, Department of Computer Science, University of Georgia, 2013
\item Outstanding Student Scholarship, Zhengzhou University, Henan, China, 2005
\end{itemize}
\end{rSection}

\end{document}
